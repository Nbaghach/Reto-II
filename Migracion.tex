% Options for packages loaded elsewhere
\PassOptionsToPackage{unicode}{hyperref}
\PassOptionsToPackage{hyphens}{url}
%
\documentclass[
]{article}
\usepackage{amsmath,amssymb}
\usepackage{iftex}
\ifPDFTeX
  \usepackage[T1]{fontenc}
  \usepackage[utf8]{inputenc}
  \usepackage{textcomp} % provide euro and other symbols
\else % if luatex or xetex
  \usepackage{unicode-math} % this also loads fontspec
  \defaultfontfeatures{Scale=MatchLowercase}
  \defaultfontfeatures[\rmfamily]{Ligatures=TeX,Scale=1}
\fi
\usepackage{lmodern}
\ifPDFTeX\else
  % xetex/luatex font selection
\fi
% Use upquote if available, for straight quotes in verbatim environments
\IfFileExists{upquote.sty}{\usepackage{upquote}}{}
\IfFileExists{microtype.sty}{% use microtype if available
  \usepackage[]{microtype}
  \UseMicrotypeSet[protrusion]{basicmath} % disable protrusion for tt fonts
}{}
\makeatletter
\@ifundefined{KOMAClassName}{% if non-KOMA class
  \IfFileExists{parskip.sty}{%
    \usepackage{parskip}
  }{% else
    \setlength{\parindent}{0pt}
    \setlength{\parskip}{6pt plus 2pt minus 1pt}}
}{% if KOMA class
  \KOMAoptions{parskip=half}}
\makeatother
\usepackage{xcolor}
\usepackage[margin=1in]{geometry}
\usepackage{longtable,booktabs,array}
\usepackage{calc} % for calculating minipage widths
% Correct order of tables after \paragraph or \subparagraph
\usepackage{etoolbox}
\makeatletter
\patchcmd\longtable{\par}{\if@noskipsec\mbox{}\fi\par}{}{}
\makeatother
% Allow footnotes in longtable head/foot
\IfFileExists{footnotehyper.sty}{\usepackage{footnotehyper}}{\usepackage{footnote}}
\makesavenoteenv{longtable}
\usepackage{graphicx}
\makeatletter
\def\maxwidth{\ifdim\Gin@nat@width>\linewidth\linewidth\else\Gin@nat@width\fi}
\def\maxheight{\ifdim\Gin@nat@height>\textheight\textheight\else\Gin@nat@height\fi}
\makeatother
% Scale images if necessary, so that they will not overflow the page
% margins by default, and it is still possible to overwrite the defaults
% using explicit options in \includegraphics[width, height, ...]{}
\setkeys{Gin}{width=\maxwidth,height=\maxheight,keepaspectratio}
% Set default figure placement to htbp
\makeatletter
\def\fps@figure{htbp}
\makeatother
\setlength{\emergencystretch}{3em} % prevent overfull lines
\providecommand{\tightlist}{%
  \setlength{\itemsep}{0pt}\setlength{\parskip}{0pt}}
\setcounter{secnumdepth}{-\maxdimen} % remove section numbering
\ifLuaTeX
  \usepackage{selnolig}  % disable illegal ligatures
\fi
\usepackage{bookmark}
\IfFileExists{xurl.sty}{\usepackage{xurl}}{} % add URL line breaks if available
\urlstyle{same}
\hypersetup{
  pdftitle={Informe t?cnico - Migraci?n neta internacional},
  pdfauthor={Nabel Baghach},
  hidelinks,
  pdfcreator={LaTeX via pandoc}}

\title{Informe t?cnico - Migraci?n neta internacional}
\author{Nabel Baghach}
\date{23 agosto 2025}

\begin{document}
\maketitle

{
\setcounter{tocdepth}{2}
\tableofcontents
}
\section{Introducci?n}\label{introduccin}

Este informe presenta un an?lisis reproducible de la migraci?n neta
internacional por pa?s y a?o a partir de datos abiertos. Se documenta la
estructura del proyecto, los objetivos, la metodolog?a y un primer
resumen de resultados.

\begin{longtable}[]{@{}
  >{\raggedright\arraybackslash}p{(\columnwidth - 8\tabcolsep) * \real{0.8070}}
  >{\raggedleft\arraybackslash}p{(\columnwidth - 8\tabcolsep) * \real{0.0351}}
  >{\raggedleft\arraybackslash}p{(\columnwidth - 8\tabcolsep) * \real{0.0526}}
  >{\raggedleft\arraybackslash}p{(\columnwidth - 8\tabcolsep) * \real{0.0526}}
  >{\raggedleft\arraybackslash}p{(\columnwidth - 8\tabcolsep) * \real{0.0526}}@{}}
\caption{Comprobaci?n de carga de datos}\tabularnewline
\toprule\noalign{}
\begin{minipage}[b]{\linewidth}\raggedright
archivo\_usado
\end{minipage} & \begin{minipage}[b]{\linewidth}\raggedleft
filas
\end{minipage} & \begin{minipage}[b]{\linewidth}\raggedleft
columnas
\end{minipage} & \begin{minipage}[b]{\linewidth}\raggedleft
anio\_min
\end{minipage} & \begin{minipage}[b]{\linewidth}\raggedleft
anio\_max
\end{minipage} \\
\midrule\noalign{}
\endfirsthead
\toprule\noalign{}
\begin{minipage}[b]{\linewidth}\raggedright
archivo\_usado
\end{minipage} & \begin{minipage}[b]{\linewidth}\raggedleft
filas
\end{minipage} & \begin{minipage}[b]{\linewidth}\raggedleft
columnas
\end{minipage} & \begin{minipage}[b]{\linewidth}\raggedleft
anio\_min
\end{minipage} & \begin{minipage}[b]{\linewidth}\raggedleft
anio\_max
\end{minipage} \\
\midrule\noalign{}
\endhead
\bottomrule\noalign{}
\endlastfoot
C:/Users/nabee/Desktop/MASTER BEHAVIOURAL DATA SCIENCE/MODULO VIII/RETO
II/Reto-II/Datos/Base\_de\_datos\_depurada/migracion\_neta\_limpio.csv &
1624 & 21 & Inf & -Inf \\
\end{longtable}

\begin{verbatim}
##  [1] "geo"                    "year"                   "immigrant_stock_value" 
##  [4] "g77_and_oecd_countries" "income_groups"          "is_country"            
##  [7] "iso3166_1_alpha2"       "unicode_region_subtag"  "iso3166_1_alpha3"      
## [10] "iso3166_1_numeric"      "landlocked"             "latitude"              
## [13] "longitude"              "main_religion_2008"     "name"                  
## [16] "un_state"               "world_4region"          "world_6region"         
## [19] "country_name"           "emigrant_stock_value"   "net_migration"
\end{verbatim}

\section{Objetivos del proyecto}\label{objetivos-del-proyecto}

\section{Objetivos del proyecto}\label{objetivos-del-proyecto-1}

\subsubsection{Objetivo principal}\label{objetivo-principal}

Analizar la evoluci?n y las diferencias en la migraci?n neta
internacional por pa?s y a?o, a partir de datos abiertos procedentes de
Gapminder/UN DESA, con el fin de identificar patrones regionales y
socioecon?micos relevantes en los flujos migratorios y contribuir a una
mejor comprensi?n de las din?micas poblacionales a nivel global.

\subsubsection{Objetivos espec?ficos}\label{objetivos-especficos}

\begin{itemize}
\tightlist
\item
  Construir una base de datos reproducible que integre los indicadores
  de inmigraci?n y emigraci?n, generando el indicador de migraci?n neta
  por pa?s y a?o.
\item
  Realizar un an?lisis descriptivo multivariante de la migraci?n neta,
  incorporando dimensiones geogr?ficas y socioecon?micas (por ejemplo,
  regiones mundiales y niveles de ingresos).
\item
  Implementar visualizaciones din?micas que permitan explorar de manera
  interactiva la evoluci?n temporal y regional de la migraci?n neta.
\item
  Elaborar un informe t?cnico reproducible que resuma los principales
  hallazgos y tendencias, integrando texto y resultados generados con R.
\item
  Dise?ar una presentaci?n acad?mica reproducible que comunique de forma
  clara y sint?tica los resultados obtenidos, incluyendo tablas,
  gr?ficos y explicaciones fundamentadas.
\end{itemize}

\#Metodolog?a (resumen)

Datos originales documentados en Datos/Base\_de\_datos\_original/.

Datasets intermedios y final en Datos/Base\_de\_datos\_depurada/:

inmigracion.csv, emigracion.csv, migracion\_neta.csv.

\#An?lisis con R (rutas relativas, c?digo comentado, resultados
reproducibles).

Carga de datos (placeholder) \# Descomenta y ajusta cuando tengas los
CSV depurados listos:

\section{library(readr);
library(dplyr)}\label{libraryreadr-librarydplyr}

\section{migr\_neta \textless-
readr::read\_csv(``Datos/Base\_de\_datos\_depurada/migracion\_neta.csv'',}\label{migr_neta---readrread_csvdatosbase_de_datos_depuradamigracion_neta.csv}

\section{show\_col\_types = FALSE)}\label{show_col_types-false}

\section{dplyr::glimpse(migr\_neta)}\label{dplyrglimpsemigr_neta}

Resultados iniciales (placeholder) \# Ejemplo de tabla/resumen cuando
cargues los datos:

\section{migr\_neta \%\textgreater\%}\label{migr_neta}

\section{group\_by(anio) \%\textgreater\%}\label{group_byanio}

\section{summarise(migracion\_neta\_total = sum(migracion\_neta, na.rm =
TRUE))}\label{summarisemigracion_neta_total-summigracion_neta-na.rm-true}

\#Conclusiones (provisionales)

Se han definido los objetivos de investigaci?n y la estructura
reproducible del proyecto.

En pr?ximas iteraciones se incorporar?n tablas y gr?ficos generados
directamente desde los datos depurados.

\#Referencias

UN DESA / Gapminder (open-numbers). Ver documentaci?n de fuente en
Datos/Base\_de\_datos\_original/.

\end{document}
